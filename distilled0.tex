\documentclass{if-beamer}
\begin{document}
	
	\title[Campus Boca do Acre]{Inglês instrumental}
	\subtitle{Class 1 }
	\author{Gilmar Gomes do Nascimento}
	\institute[IFAM]{
		Instituto Federal do Amazonas\\
		Campus Boca do Acre
	}
	\date{\today}
	\logo{
		\includegraphics[scale=0.0085]{logo.jpg}
	}
	\subject{Inglês Instrumental} % metadata
	
	\graphicspath{{figuras/}}
	%---------------------------------------------------------------------------------
	\begin{frame}
		\titlepage
	\end{frame}
	
\begin{frame}{Cognatos}
			\begin{exampleblock}{\textbf{Cognatos Verdadeiros}}
				\begin{itemize}
					\item \textbf{informação} $\rightarrow$ \textbf{information}
					\item \textbf{tecnologia} $\rightarrow$ \textbf{technology}
					\item \textbf{documento} $\rightarrow$ \textbf{document}
					\item \textbf{aplicativo} $\rightarrow$ \textbf{application}
					\item \textbf{interface} $\rightarrow$ \textbf{interface}
				\end{itemize}
			\end{exampleblock}
		\end{frame}
		
			\begin{frame}{\textbf{Falsos Cognatos}}
				\begin{itemize}
					\item \textbf{pretender} $\rightarrow$ \textbf{intend} \quad (not ``pretend")
					\item \textbf{assumir} $\rightarrow$ \textbf{presume} \quad (not ``assume")
					\item \textbf{enrolar} $\rightarrow$ \textbf{wrap} \quad (not ``enroll")
					\item \textbf{esquisito} $\rightarrow$ \textbf{weird} \quad (not ``exquisite")
					\item \textbf{passagem} $\rightarrow$ \textbf{ticket} \quad (sometimes!)
				\end{itemize}
			\end{frame}
		
	
	\begin{frame}{Computer Science - Distilled}
		\centering
		\includegraphics[scale=0.15]{distilledbook.jpg}	
	\end{frame}
	
	\begin{frame}{Distilled?}
		\centering
		\includegraphics[scale=0.35]{distilled-meaning.png}
	\end{frame}
	
	\begin{frame}{Phrasal verbs}
		\begin{exampleblock}{Phrasal verbs}
		Os phrasal verbs (\textbf{verbos frasais}) são verbos que vêm acompanhados por preposições ou advérbios. \\
		
		Com o acréscimo de uma preposição ou de um advérbio, o sentido do verbo original pode mudar completamente. \\
		
		Por isso, os verbos frasais não podem ser traduzidos literalmente, palavra por palavra. 
		\end{exampleblock}
	\end{frame}
	
	\begin{frame}{4 phrasal verbs with `BOOK'}
	\begin{itemize}
		\item Book in
		\item Book into
		\item Book out
		\item Book up
	\end{itemize}
	\end{frame}
	\begin{frame}{Book in}
		Make a reservation in advance \\
		(Separable [obligatory] | British English) \\
		 Example: I'll BOOK us IN at the Intercontinental. \\
		Check in at a hotel
		(Separable [obligatory] | British English) \\
		 Example: WE took a taxi from the airport to the hotel and BOOKED IN.
	\end{frame}
	
	\begin{frame}{Book into}
	Make a reservation in advance \\
	(Separable [obligatory] | British English)
	\\ Example: I`ve BOOKED us INTO a hotel in the centre of town for three nights.
	Check in at a hotel
	\\(Separable [obligatory] | British English) \\
	 Example: We BOOKED INTO the first hotel we could find.
	\end{frame}
	
	\begin{frame}{Book out}
		Leave a place in a hurry \\
		(Intransitive | International English) \\
		 Example: I don't like the look of the people arriving- let's BOOK OUT.
	\end{frame}
	
	\begin{frame}{Book up}
		Reserve \\
		(Intransitive | International English)
		 Example: The flight's fully BOOKED UP; I'll have to go the following day.
	\end{frame}
	
	\begin{frame}{Other meaning}
		\justifying
		\textbf{Overbooking} is the practice of selling or reserving more seats or accommodations than the available capacity of a service, commonly seen in the travel and hospitality industries. This often occurs to compensate for expected cancellations or no-shows, ensuring that businesses maximize their occupancy or capacity. In cases of overbooking, passengers may be denied boarding even if they have met all requirements, leading to potential inconveniences.
	\end{frame}
	
	\begin{frame}[allowframebreaks]{Compound words}
		\begin{itemize}
			\item Bookworm: Pessoa que adora ler (literalmente: "traça de livro").
			\item Bookkeeper / Bookkeeping: Contador / Contabilidade (a pessoa que mantém os livros de registros).
			\item Booklet: Um livrinho ou folheto (sufixo -let indica pequeno).
			\item Bookcase / Bookshelf: Estante para livros.
			
			\framebreak
			
			\item Bookmark: Marcador de páginas (ou o favorito no navegador).
			\item Bookend: Suporte para sustentar livros na estante.
			\item Bookfair: Feira do livro.
			\item Bookmaker (ou ``Bookie"): Agente ou casa de apostas (que faz o livro das probabilidades).
			\item Booksmart: Inteligente no sentido acadêmico/theórico (em oposição a street smart).
		\end{itemize}
	\end{frame}
	
	\begin{frame}{Back cover}
		\centering
		\includegraphics[scale=0.12]{backcover.jpeg}
	\end{frame}
	

	\begin{frame}{Book spine}
	\huge
		Why is too slow read on paper book? Because there is a book spine. 
	\end{frame}
	
	\begin{frame}{Computers}
	\begin{center}
		\huge
		Any sufficiently advanced technology is indistinguishable from magic.
	\end{center}
		\begin{flushright}
			Arthur C. Clarke
		\end{flushright}
		
	\end{frame}
	
	\begin{frame}{Programming}
		\begin{center}
			\huge
			When someone says: ``I want a programming language in which I need only say what I wish done", give him a lollipop.
		\end{center}
		\begin{flushright}
			Alan J. Perlis
		\end{flushright}
		
	\end{frame}
		
	\begin{frame}{Meme}
		\centering
		\includegraphics[scale=0.3]{keepcalm.jpg}
	\end{frame}
%	\begin{frame}{Git}
%		\begin{figure}
%			\centering
%			\includegraphics[scale=0.3]{git-site1.png}
%		\end{figure}
%	\end{frame}
	
	

	
%	
%	\begin{frame}
%		
%	\end{frame}	
%	
%	\begin{frame}
%		
%	\end{frame}
%	\begin{frame}
%		
%	\end{frame}
%	\begin{frame}
%		
%	\end{frame}
	\begin{frame}[allowframebreaks]{Referências Bibliográficas}
		\justifying 
		\small 
		
		\begin{thebibliography}{9}
			
				\bibitem{muniz2026}
				MUNIZ, Carla. \textbf{Phrasal Verbs}. Disponível em: \url{https://www.todamateria.com.br/phrasal-verbs/}. Acesso em: 10 fev. 2026.
				
				\bibitem{usingenglish2026}
				\textbf{Phrasal Verbs with 'Book'.} UsingEnglish.com. Disponível em: \url{https://www.usingenglish.com/reference/phrasal-verbs/book.html}. Acesso em: 10 fev. 2026.
				
				\bibitem{ferreira2017}
				FERREIRA FILHO, Wladston. \textbf{Computer science distilled: learn the art of solving computational problems}. Estados Unidos: Code Energy, 2017.
				
			
			
			%\framebreak
%			
%			\bibitem{fernandes-texto}
%			FERNANDES, M.; DIANA, D. \textbf{Texto Injuntivo}. Disponível em: \url{https://www.todamateria.com.br/texto-injuntivo/}. Acesso em: 7 ago. 2025.
%			
%			%	\framebreak
%			
%			\bibitem{mdn-javascript}
%			MDN. \textbf{JavaScript}. Disponível em: \url{https://developer.mozilla.org/en-US/docs/Web/JavaScript}. Acesso em: 7 ago. 2025.
%			
%			%	\framebreak
%			
%			\bibitem{rossini-anotacoes}
%			ROSSINI, M. C. \textbf{5 razões científicas para anotar coisas no papel em vez do celular}. Disponível em: \url{https://super.abril.com.br/ciencia/5-razoes-cientificas-para-anotar-coisas-no-papel-em-vez-do-celular/}. Acesso em: 7 ago. 2025.
			
		\end{thebibliography}
		

	\end{frame}	
	
	%\end{frame}
\end{document}
