\documentclass{if-beamer}
\begin{document}
	
	\title[Campus Boca do Acre]{Inglês instrumental}
	\subtitle{Class 0 }
	\author{Gilmar Gomes do Nascimento}
	\institute[IFAM]{
		Instituto Federal do Amazonas\\
		Campus Boca do Acre
	}
	\date{\today}
	\logo{
		\includegraphics[scale=0.0085]{logo.jpg}
	}
	\subject{Inglês Instrumental} % metadata
	
	\graphicspath{{figuras/}}
	%---------------------------------------------------------------------------------
	\begin{frame}
		\titlepage
	\end{frame}
	
	%\begin{frame}{Formação Acadêmica}
	%\begin{itemize}
	%\item Ciência da Computação - UFT   2009 -- 2014
	%\item Especialização Lato Senso - ESAB - 2019
	%\item Especialização Lato Senso - Focus - 2021
	%\item Mestrado em Ciência da Computação - UTFPR - 2024 - atualmente 
	%\end{itemize}
	%
	%\end{frame}
	%
	%\begin{frame}{Atuação Profissional}
	%\begin{itemize}
	%\item Instrutor de Informática  - SENAC - TO  - 2014, 2015
	%\item Professor de Informática - Colégio da Polícia Militar - TO - 2015 - 2016
	%\item Analista de Suporte -  Justiça Federal  - TO - 2017 
	%\item Professor substituto EBTT - IFTO - Campus Colinas do Tocantins - 2017 - 2019
	%\item Professor EBTT - IFAM - Campus Boca do Acre - 2021 - atualmente
	%\end{itemize}
	%\end{frame}
	\begin{frame}
	
	\end{frame}	
	
	\begin{frame}
		
	\end{frame}
	\begin{frame}
		
	\end{frame}
	\begin{frame}
		
	\end{frame}
	\begin{frame}[allowframebreaks]{Referências Bibliográficas}
		\justifying % Justifica o texto
		\small % Reduz um pouco o tamanho da fonte
		
		\begin{thebibliography}{9}
			
			\bibitem{wikipedia-informatics}
			Wikipedia contributors. (2025, December 7). Informatics. In Wikipedia, The Free Encyclopedia. Retrieved 18:03, January 29, 2026, from \url{https://en.wikipedia.org/w/index.php?title=Informatics&oldid=1326169972}. Acesso em: 29 jan. 2026
			
			%	\framebreak % Quebra para o próximo frame/slide
			
			\bibitem{wikipedia-linguagem}
			LINGUAGEM DE PROGRAMAÇÃO. In: \textbf{Wikipédia, a enciclopédia livre}. Flórida: Wikimedia Foundation, 2025. Disponível em: \url{https://pt.wikipedia.org/w/index.php?title=Linguagem_de_programação&oldid=69620271}. Acesso em: 23 fev. 2025.
			
			%		\framebreak
			
			\bibitem{microsoft-extensoes}
			MICROSOFT. \textbf{Extensões de nome de arquivo comuns no Windows}. Disponível em: \url{https://support.microsoft.com/pt-br/windows/extensões-de-nome-de-arquivo-comuns-no-windows-da4a4430-8e76-89c5-59f7-1cdbbc75cb01}. Acesso em: 7 ago. 2025.
			
			\framebreak
			
			\bibitem{fernandes-texto}
			FERNANDES, M.; DIANA, D. \textbf{Texto Injuntivo}. Disponível em: \url{https://www.todamateria.com.br/texto-injuntivo/}. Acesso em: 7 ago. 2025.
			
			%	\framebreak
			
			\bibitem{mdn-javascript}
			MDN. \textbf{JavaScript}. Disponível em: \url{https://developer.mozilla.org/en-US/docs/Web/JavaScript}. Acesso em: 7 ago. 2025.
			
			%	\framebreak
			
			\bibitem{rossini-anotacoes}
			ROSSINI, M. C. \textbf{5 razões científicas para anotar coisas no papel em vez do celular}. Disponível em: \url{https://super.abril.com.br/ciencia/5-razoes-cientificas-para-anotar-coisas-no-papel-em-vez-do-celular/}. Acesso em: 7 ago. 2025.
			
		\end{thebibliography}
	\end{frame}	
	
	%\end{frame}
\end{document}
