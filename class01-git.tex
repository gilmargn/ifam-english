\documentclass{if-beamer}
\begin{document}
	
	\title[Campus Boca do Acre]{Inglês instrumental}
	\subtitle{Class 4 }
	\author{Gilmar Gomes do Nascimento}
	\institute[IFAM]{
		Instituto Federal do Amazonas\\
		Campus Boca do Acre
	}
	\date{\today}
	\logo{
		\includegraphics[scale=0.0085]{logo.jpg}
	}
	\subject{Inglês Instrumental} % metadata
	
	\graphicspath{{figuras/}}
	%---------------------------------------------------------------------------------
	\begin{frame}
		\titlepage
	\end{frame}
	
	%\begin{frame}{Formação Acadêmica}
	%\begin{itemize}
	%\item Ciência da Computação - UFT   2009 -- 2014
	%\item Especialização Lato Senso - ESAB - 2019
	%\item Especialização Lato Senso - Focus - 2021
	%\item Mestrado em Ciência da Computação - UTFPR - 2024 - atualmente 
	%\end{itemize}
	%
	%\end{frame}
	%
	%\begin{frame}{Atuação Profissional}
	%\begin{itemize}
	%\item Instrutor de Informática  - SENAC - TO  - 2014, 2015
	%\item Professor de Informática - Colégio da Polícia Militar - TO - 2015 - 2016
	%\item Analista de Suporte -  Justiça Federal  - TO - 2017 
	%\item Professor substituto EBTT - IFTO - Campus Colinas do Tocantins - 2017 - 2019
	%\item Professor EBTT - IFAM - Campus Boca do Acre - 2021 - atualmente
	%\end{itemize}
	%\end{frame}
	
	
	\begin{frame}{Chocolatey}
		\begin{figure}
			\centering
			\includegraphics[scale=0.3]{chocolatey.png}
		\end{figure}
	\end{frame}
	
	\begin{frame}{Community}
		\begin{figure}
			\centering
			\includegraphics[scale=0.3]{chocolatey-community.png}
		\end{figure}
	\end{frame}
	
	\begin{frame}{How to install}
		\begin{figure}
			\centering
			\includegraphics[scale=0.3]{installing.png}
		\end{figure}
	\end{frame}

	\begin{frame}{How to install}
		\begin{figure}
			\centering
			\includegraphics[scale=0.7]{obs-installing.png}
		\end{figure}
	\end{frame}	
	
	\begin{frame}{Command Line Interface}
		\centering
		\includegraphics[scale=0.34]{powershell.png}
	\end{frame}
	
	\begin{frame}%{Basic PowerShell Commands}
		
		\begin{tabular}{|l|l|l|}
			\hline
			\textbf{Action} & \textbf{Command} & \textbf{Example} \\
			\hline
			List files & \texttt{ls} or \texttt{Get-ChildItem} & \texttt{ls} \\
			Change directory & \texttt{cd} or \texttt{Set-Location} & \texttt{cd C:\textbackslash Users} \\
			Create folder & \texttt{mkdir} or \texttt{md} & \texttt{mkdir NewFolder} \\
			Current directory & \texttt{pwd} or \texttt{Get-Location} & \texttt{pwd} \\
			Copy file & cp or Copy-Item & cp file.txt backup.txt \\
			Delete file & \texttt{del} or \texttt{Remove-Item} & \texttt{del file.txt} \\
			Delete folder & \texttt{rmdir} or \texttt{Remove-Item} & \texttt{rmdir Folder} \\
			Clear screen & \texttt{cls} or \texttt{Clear-Host} & \texttt{cls} \\
			Create file & \texttt{ni} or \texttt{New-Item} & \texttt{ni file.txt} \\
			\hline
		\end{tabular}
		
	\end{frame}
	
	\begin{frame}{XAMPP}
		\centering
		\includegraphics[scale=0.5]{xampp.png}
	\end{frame}
	
	\begin{frame}{MariaDB}
		\centering
		\includegraphics[scale=0.7]{mariadb1.png}
	\end{frame}
	\begin{frame}{Let's create and populate a database}
	\begin{itemize}
		\item Log in
		\item show databases
		\item create database
		\item create table
		\item insert into table
		\item show tables
		\item show content
		\item export database
	\end{itemize}
	\end{frame}
	
	\begin{frame}{Git}
		\begin{itemize}
			\item Git init
			\item Git add
			\item Git commit
			\item Git branch -M main
			\item Git push -u origin 
			\item To show on github
		\end{itemize}
	\end{frame}
	\begin{frame}[allowframebreaks]{Referências Bibliográficas}
		\justifying % Justifica o texto
		\small % Reduz um pouco o tamanho da fonte
		
		\begin{thebibliography}{9}
			
			\bibitem{wikipedia-informatics}
			Wikipedia contributors. (2025, December 7). Informatics. In Wikipedia, The Free Encyclopedia. Retrieved 18:03, January 29, 2026, from \url{https://en.wikipedia.org/w/index.php?title=Informatics&oldid=1326169972}. Acesso em: 29 jan. 2026
			
			%	\framebreak % Quebra para o próximo frame/slide
			
			\bibitem{wikipedia-linguagem}
			LINGUAGEM DE PROGRAMAÇÃO. In: \textbf{Wikipédia, a enciclopédia livre}. Flórida: Wikimedia Foundation, 2025. Disponível em: \url{https://pt.wikipedia.org/w/index.php?title=Linguagem_de_programação&oldid=69620271}. Acesso em: 23 fev. 2025.
			
			%		\framebreak
			
			\bibitem{microsoft-extensoes}
			MICROSOFT. \textbf{Extensões de nome de arquivo comuns no Windows}. Disponível em: \url{https://support.microsoft.com/pt-br/windows/extensões-de-nome-de-arquivo-comuns-no-windows-da4a4430-8e76-89c5-59f7-1cdbbc75cb01}. Acesso em: 7 ago. 2025.
			
			\framebreak
			
			\bibitem{fernandes-texto}
			FERNANDES, M.; DIANA, D. \textbf{Texto Injuntivo}. Disponível em: \url{https://www.todamateria.com.br/texto-injuntivo/}. Acesso em: 7 ago. 2025.
			
			%	\framebreak
			
			\bibitem{mdn-javascript}
			MDN. \textbf{JavaScript}. Disponível em: \url{https://developer.mozilla.org/en-US/docs/Web/JavaScript}. Acesso em: 7 ago. 2025.
			
			%	\framebreak
			
			\bibitem{rossini-anotacoes}
			ROSSINI, M. C. \textbf{5 razões científicas para anotar coisas no papel em vez do celular}. Disponível em: \url{https://super.abril.com.br/ciencia/5-razoes-cientificas-para-anotar-coisas-no-papel-em-vez-do-celular/}. Acesso em: 7 ago. 2025.
			
		\end{thebibliography}
	\end{frame}	
	
	%\end{frame}
\end{document}
