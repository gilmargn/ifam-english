\documentclass[portuguese,10pt,a4paper]{article}
\usepackage[T1]{fontenc}
\usepackage{graphicx}
\usepackage{babel}
\usepackage{hyperref}
\usepackage{multicol}
\title{Smart city}
\author{Gilmar Gomes do Nascimento}
\begin{document}
	\begin{center}
		\includegraphics[height=1.5cm]{dco.png}
		\hfill
		\includegraphics[height=1.5cm]{qrcode-smartcity}
		\hfill
		\includegraphics[height=1.5cm]{ifam}
%		\vspace{0.5cm}
		\hrule
%		\vspace{1cm}
	\end{center}
	%	\maketitle
	\section*{Smart cities}\label{vivaldi-takes-a-stand-keep-browsing-human}
	
	\begin{multicols}{2}
		
		A smart city refers to an urban area that leverages digital technologies to enhance the quality of life, optimize resource use, and ensure sustainability.
		It utilizes cutting-edge digital technology tools to improve infrastructure, governance, and public services [1], [2], [3]. 
		
		Smart cities aim to create sustainable urban ecosystems that balance technological innovation with social, environmental, and economic 		objectives [4], [5].
		
		The backbone of smart cities lies in digital technologies, such as IoT, AI, cloud computing, and big data. These technologies enable real-time data collection and 	analysis, facilitating improved decision-making and city management [6], [7]. 
		
		IoT devices monitor various city operations, from traffic management to energy
		consumption, while AI helps predict urban challenges like congestion and air pollution [8], [9]. Smart cities contribute significantly to the 		digital economy by fostering innovation and creating new business models. The
		integration of digital technologies enhances productivity, stimulates local businesses, and attracts investment [10] 
		
		Digital technologies offer numerous benefits to smart cities residents, such as automated healthcare, transportation, and education services [12], [13] ; energy-efficient buildings [5], [10]; and data-driven transparent 		governance [14], [15]. The implementation of digital technologies in smart cities enhances the overall quality of life and contributes to 	a more sustainable digital economy. 
		
		However, the deployment of digital technologies in smart cities presents
		various challenges, including privacy and security concerns, digital inequality, ethical and social issues, and environmental sustainability [2], [16], [17]. 
		
		This paper examines these challenges through the broader aspects of inclusivity, human-centric 	development, and sustainability related to
		the use of digital technologies in smart cities. 
		
		It proposes viable solutions, presented as strategic recommendations, to address these challenges and promote a human-centric, all-inclusive, and sustainable digital economy.
		
		We propose a set of strategic recommendations for stakeholders in the digital economy to foster more inclusive, human-centric, and sustainable urban environments in smart	cities. For instance, to ensure inclusiveness,
		governments can implement robust data protection laws, while businesses and Public-Private Partnerships (PPPs) should focus on user-centric designs and public engagement.
		
		Moreover, a human-centric approach prioritizes ethical AI use, inclusive public spaces, and participatory feedback mechanisms that reflect 		community needs.
	\end{multicols}
	\begin{figure}
		\centering
		\includegraphics[width=1.0\textwidth]{smartcities-concept}
	\end{figure}
	\begin{multicols}{2}	
		Besides, sustainability goals can be achieved through interdisciplinary research, green infrastructure, circular economic initiatives, and transparency in environmental claims. 
		
		Collaborative efforts among smart city 	stakeholders, including governments,
		businesses, civil societies, and international 	organizations, are pivotal in integrating long-term planning, ethical governance, and community resilience into smart city projects, making them truly inclusive, human-centered, and sustainable.
		
		In this article, we adopted a literature review methodology to search, analyze, and consolidate key challenges posed by digital technologies in smart cities. Our contribution lies in bridging research gaps 	by exploring the interplay between smart cities, the challenges of digital technologies
		and digital economy. Additionally, we propose strategic recommendations to
		address these challenges, emphasizing humanity, inclusivity, and sustainability to ensure equitable participation in the smart
		ecosystem. Therefore, we formulated the following.
		\subsection*{Research Questions (RQs):}
		
		\begin{enumerate}
			\item How do digital technologies in smart cities pose challenges in terms of inclusivity, humanity and sustainability?
			\item  What strategic recommendations can help stakeholders address smart city	challenges while promoting inclusivity, humanity, and sustainability
		\end{enumerate}
	\end{multicols}
	

\end{document}