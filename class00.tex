\documentclass{if-beamer}
\begin{document}
	
	\title[Campus Boca do Acre]{Inglês instrumental}
	\subtitle{Class 0 }
	\author{Gilmar Gomes do Nascimento}
	\institute[IFAM]{
		Instituto Federal do Amazonas\\
		Campus Boca do Acre
	}
	\date{\today}
	\logo{
		\includegraphics[scale=0.0085]{logo.jpg}
	}
	\subject{Inglês Instrumental} % metadata
	
	\graphicspath{{figuras/}}
	%---------------------------------------------------------------------------------
	\begin{frame}
		\titlepage
	\end{frame}
	
	%\begin{frame}{Formação Acadêmica}
	%\begin{itemize}
	%\item Ciência da Computação - UFT   2009 -- 2014
	%\item Especialização Lato Senso - ESAB - 2019
	%\item Especialização Lato Senso - Focus - 2021
	%\item Mestrado em Ciência da Computação - UTFPR - 2024 - atualmente 
	%\end{itemize}
	%
	%\end{frame}
	%
	%\begin{frame}{Atuação Profissional}
	%\begin{itemize}
	%\item Instrutor de Informática  - SENAC - TO  - 2014, 2015
	%\item Professor de Informática - Colégio da Polícia Militar - TO - 2015 - 2016
	%\item Analista de Suporte -  Justiça Federal  - TO - 2017 
	%\item Professor substituto EBTT - IFTO - Campus Colinas do Tocantins - 2017 - 2019
	%\item Professor EBTT - IFAM - Campus Boca do Acre - 2021 - atualmente
	%\end{itemize}
	%\end{frame}
	\begin{frame}{Ementa}
		\begin{figure}
			\includegraphics[scale=0.46]{ementa}
		\end{figure}
		
		
	\end{frame}
	%------------------------------------------
	\begin{frame}
		\frametitle{Objetivo Geral}
		Capacitar o aluno a ler textos da área de informática. 
		\end{frame}
	%----------------------------------------------------------------------
		
	\begin{frame}{Objetivos Específicos}
		\begin{itemize}
			\item Orientar sobre a utilização de estratégias de leitura e noções da estrutura da língua inglesa.  
			\item Propor aquisição de vocabulário técnico.
			
		\end{itemize}
	\end{frame}
	%------------------------------------------------------------------------------------
\begin{frame}{Conteúdo programático}
    \begin{enumerate}
        \item Leitura e compreensão escrita (Reading Strategies)
            \begin{itemize}
                \item Objetivos da leitura: o texto técnico  
                \item Skimming/Scanning. 
                \item Cognatos e falsos cognatos; estrangeirismos. 
                \item Leitura para reconhecimento de tema central, ideia central e keywords. 
                \item Grupos nominais, siglas e acrônimos 
            \end{itemize}
        \item Compreensão oral:
            \begin{itemize}
                \item Conceito, identificação e função de gêneros textuais. 
                \item Inferências
                \item Rotinas conversacionais
            \end{itemize} 
    \end{enumerate}
\end{frame}

\begin{frame}{Conteúdo programático (continuação)}
    \begin{enumerate}
        \setcounter{enumi}{2}  
        \item Aspectos léxico-gramaticais da língua Inglesa
            \begin{itemize}
                \item Formação de palavras (afixos)
                \item Vocabulário técnico
            \end{itemize}
        \item Produção escrita
            \begin{itemize}
                \item Marcadores discursivos
                \item Diferentes tipos de texto
                \item Conjunções 
                \item Referência pronominal
            \end{itemize}
    \end{enumerate}
\end{frame}	%----------------------------------------------------------------------------------
\begin{frame}{Esclarecimento importante}
    \textbf{\large O que esta disciplina NÃO é:}
    
    \begin{itemize}
        \item \textbf{Não é} um curso de conversação em inglês
        \item \textbf{Não é} um treinamento de compreensão auditiva
        \item \textbf{Não é} um curso de pronúncia ou redução de sotaque
    \end{itemize}
    
    \vspace{0.8cm}
    
    \textbf{\large O que esta disciplina É:}
    
    \begin{itemize}
        \item \textbf{É} um curso focado em \textbf{leitura técnica}
        \item \textbf{É} uma disciplina de \textbf{interpretação de textos científicos}
        \item \textbf{É} um treinamento para \textbf{compreensão de documentação de código}       
    \end{itemize}
    
    \vspace{0.8cm}
    
    \begin{center}
        \fcolorbox{blue}{blue!5}{
            \begin{minipage}{0.9\textwidth}
                \centering
                \small
                \textbf{Justificativa:} Em contextos técnicos/acadêmicos, a habilidade de leitura  é prioritária para acesso a documentação, artigos e recursos em inglês.
            \end{minipage}
        }
    \end{center}
\end{frame}

\begin{frame}{Tesauro Técnico - Sua Ferramenta Pessoal}
    
    \begin{center}
        \textbf{\LARGE Por que escrever manualmente?}
    \end{center}
    
    \vspace{0.5cm}
    
    \begin{columns}[T]
        \begin{column}{0.32\textwidth}
            \begin{block}{Memória}
                \centering
                \textbf{Mão ↔ Cérebro} \\
                \small
                Escrever ativa circuitos neurais que reforçam a memória de longo prazo.
            \end{block}
        \end{column}
        
        \begin{column}{0.32\textwidth}
            \begin{block}{Compreensão}
                \centering
                \textbf{Processamento profundo} \\
                \small
                Para definir com suas palavras, você precisa realmente entender o conceito.
            \end{block}
        \end{column}
        
        \begin{column}{0.32\textwidth}
            \begin{block}{Autonomia}
                \centering
                \textbf{Recurso pessoal} \\
                \small
                Você constrói um recurso adaptado ao seu contexto e necessidades.
            \end{block}
        \end{column}
    \end{columns}
    
\end{frame}

\begin{frame}{LLMs}

    \begin{center}
        \begin{minipage}{0.8\textwidth}
            {Na era dos LLMs e buscas instantâneas...}
                \centering
                \textbf{O processo manual não é anacrônico — é estratégico.} \\
                \small Desenvolve paciência, profundidade e reflexão, competências cada vez mais raras.
            
        \end{minipage}
    \end{center}

\end{frame}



	\begin{frame}{Pedagogia e Métodos de avaliação}
	\begin{block}{Avaliação da aprendizagem}
		Conforme o artigo 34° da Resolução N° 6 de 20 de setembro de 2012, a
		avaliação da aprendizagem dos estudantes visa à sua progressão para o 
		alcance 	do perfil profissional de conclusão, sendo contínua e cumulativa, 
		com prevalência 		dos aspectos qualitativos sobre os quantitativos, bem 
		como dos resultados ao 	longo do processo sobre os de eventuais provas 
		finais.
	\end{block}
\end{frame}
%----------------------------------------------------------------------------------------
\begin{frame}{Avaliação de aprendizagem}
	\begin{block}{Pesos}
		Análise dos algoritmos e atividades resolvidas. Os algoritmos e atividades terão pesos(notas) diferentes. 
	\end{block}
	\begin{block}{Constâncias}
		Quanto maior a constância, maior a chance de aprendizado e notas boas.
	\end{block}
	\begin{block}{Uso de IA}
		Sei que é praticamente impossível a proibição de IA e de outras colas, contudo, o algoritmo pode ser questionado linha por linha. Se o estudante conseguir explicar o funcionamento, não há problemas, contudo, não será aceito, estratégias de algoritmos não ensinadas até o momento.
	\end{block}
\end{frame}
%----------------------------------------------------------------------------------------
\begin{frame}{}
	\begin{block}{Avaliação do sistema educacional}
		A avaliação do rendimento acadêmico deve ser feita por componente
		curricular/disciplina, abrangendo simultaneamente os aspectos de frequência e de 	aproveitamento de conhecimentos.
	\end{block}
	\begin{block}{Nota}
		O registro da avaliação da aprendizagem deverá ser expresso em nota e
		obedecerá a uma escala de valores de 0 a 10 (zero a dez), cuja pontuação mínima
		para promoção seguirá os critérios estabelecidos na organização didática do
		IFAM. Atualmente, conforme a Resolução Nº 94 CONSUP/IFAM de 23/12/2015 a
		pontuação mínima é de 6,0 (seis) por disciplina.	
	\end{block}
\end{frame}
%-------------------------------------------------------------------------------------------------
\begin{frame}{Instrumentos avaliativos}
	03 (três) instrumentos avaliativos, sendo 01 (um) escrito por módulo
	letivo para a Educação Profissional Técnica de Nível Médio nas Formas
	Subsequente e Concomitante, e na Forma Integrada à Modalidade de Educação
	de Jovens e Adultos – EJA/EPT
\end{frame}

\begin{frame}{Correlação entre Inglês e Informática}
\large
\centering
De acordo o Ortiz (2007) na área de tecnologia, o inglês é como o idioma base
para a sintaxe das linguagens de programação, são utilizadas palavras em inglês
automaticamente ao programar, como o ``if", ``while" ~ e `` for"; além disso, a maioria 
dos materiais de referência são em inglês. Por isso, é muito importante saber inglês 
até mesmo para desenvolver os algoritmos.
\end{frame}

\begin{frame}{BIOS}
	\huge
	\centering
	\textbf{B} asic \\
	\textbf{I} nput \\
	\textbf{O} utput \\
	\textbf{S} ystem
	
\end{frame}
\begin{frame}{Implementations}
	Most BIOS implementations are specifically designed to work with a particular computer or motherboard model, by interfacing with various devices especially system chipset. Originally, BIOS firmware was stored in a ROM chip on the PC motherboard. In later computer systems, the BIOS contents are stored on flash memory so it can be rewritten without removing the chip from the motherboard. 
\end{frame}
\begin{frame}{Features}
	This allows easy, end-user updates to the BIOS firmware so new features can be added or bugs can be fixed, but it also creates a possibility for the computer to become infected with BIOS rootkits. Furthermore, a BIOS upgrade that fails could brick the motherboard.
\end{frame}

\begin{frame}{UEFI}
\centering
\includegraphics[scale=0.8]{uefi}
\end{frame}

\begin{frame}{UEFI}
	\large
	Unified Extensible Firmware Interface (UEFI) is a specification for the firmware architecture of a computing platform. When a computer is powered on, the UEFI implementation is typically the first that runs, before starting the operating system. 	
\end{frame}
\begin{frame}{Interface}
	\begin{itemize}
		\item Keyboard
		\item Mouse
	\end{itemize}
	
\end{frame}
\begin{frame}{Input}
	Keyboard input refers to the process of sending text, commands, and data from a keyboard to a computer or device, where each key press generates a digital signal that represents a character or function, handled by drivers and layouts to display it on screen or perform an action, with various methods including physical keyboards, on-screen keyboards, and touch input, and specific software events for developers. 
	
\end{frame}
\begin{frame}{Keyboards}
	\centering
	\includegraphics[scale=0.5]{keyboard}
\end{frame}
\begin{frame}{Arrow Keys - Teclas de direção}
	\centering
	\begin{tabular}{c|c}
		up arrow & seta para cima \\
		down arrow & seta para baixo \\
		left arrow & seta para esquerda \\
		right arrow & seta para direita \\
		\hline
		
	\end{tabular}
\end{frame}
\begin{frame}
	\begin{exampleblock}{\textbf{Common Uses in Documentation}}
		\begin{itemize}
			\item \textbf{``Use the arrow keys to navigate the menu"} \\
			\textit{``Use as setas direcionais para navegar no menu"}
			
			\item \textbf{``Press the up arrow to see previous commands"} \\
			\textit{``Pressione a seta para cima para ver comandos anteriores"}
			
			\item \textbf{``Hold Shift and use arrow keys to select text"} \\
			\textit{``Mantenha \textit{Shift} pressionado e use as setas para selecionar texto"}
		\end{itemize}
	\end{exampleblock}
\end{frame}

\begin{frame}{Arrow Keys in Real Documentation}
	
	\begin{block}{Example 1: Text Editor}
		\textbf{``Navigate through the document using the arrow keys"} \\
		\textit{``Navegue pelo documento usando as setas direcionais"}
	\end{block}
	
	\begin{block}{Example 2: Command Line/Terminal}
		\textbf{``Press the up arrow to recall the previous command"} \\
		\textit{``Pressione a seta para cima para recuperar o comando anterior"}
	\end{block}
\end{frame}
\begin{frame}
	\begin{block}{Example 3: Spreadsheet Software}
		\textbf{``Move between cells with the arrow keys"} \\
		\textit{``Mova-se entre células com as setas direcionais"}
	\end{block}
	
	\begin{block}{Example 4: Game Controls}
		\textbf{``Use WASD or arrow keys for movement"} \\
		\textit{``Use WASD ou as setas para movimento"}
	\end{block}
	
\end{frame}
\begin{frame}
	\begin{alertblock}{\textbf{Important Note}}
		In some contexts, especially older documentation or specific software, \\
		you might see: \textbf{cursor keys} = \textbf{teclas do cursor} \\
		(Same meaning as arrow keys)
	\end{alertblock}
\end{frame}

\begin{frame}{Keyboard Terminology - Basics}
	\begin{columns}[T]
		\begin{column}{0.48\textwidth}
			\begin{block}{\textbf{Hardware Components}}
				\begin{itemize}
					\item \textbf{key} $\rightarrow$ \textbf{tecla}
					\item \textbf{keyboard} $\rightarrow$ \textbf{teclado}
					\item \textbf{keycap} $\rightarrow$ \textbf{capa da tecla}
					\item \textbf{keypad} $\rightarrow$ \textbf{teclado numérico}
					\item \textbf{keyswitch} $\rightarrow$ \textbf{interruptor da tecla}
					\item \textbf{layout} $\rightarrow$ \textbf{disposição/arranjo}
				\end{itemize}
			\end{block}
		\end{column}
		
		\begin{column}{0.48\textwidth}
			\begin{block}{\textbf{Actions}}
				\begin{itemize}
					\item \textbf{type} $\rightarrow$ \textbf{digitar/datilografar}
					\item \textbf{keystroke} $\rightarrow$ \textbf{pressionamento de tecla}
					\item \textbf{press} $\rightarrow$ \textbf{pressionar}
					\item \textbf{hold} $\rightarrow$ \textbf{segurar/manter pressionado}
					\item \textbf{release} $\rightarrow$ \textbf{soltar}
					\item \textbf{tap} $\rightarrow$ \textbf{bater/acionar rapidamente}
				\end{itemize}
			\end{block}
		\end{column}
	\end{columns}
	
	\vspace{0.5cm}
	
	\begin{center}
		\fcolorbox{blue}{blue!5}{
			\begin{minipage}{0.9\textwidth}
				\centering
				\textbf{Context:} "Press the \textbf{Enter} key" = "Pressione a tecla \textbf{Enter}"
			\end{minipage}
		}
	\end{center}
\end{frame}

\begin{frame}{Special Keys and Modifiers}
	\begin{columns}[T]
		\begin{column}{0.32\textwidth}
			\begin{block}{\textbf{Modifier Keys}}
				\begin{itemize}
					\small
					\item \textbf{Shift} $\rightarrow$ \textbf{Shift}
					\item \textbf{Ctrl (Control)} $\rightarrow$ \textbf{Ctrl}
					\item \textbf{Alt (Alternate)} $\rightarrow$ \textbf{Alt}
					\item \textbf{Cmd (Command)} $\rightarrow$ \textbf{Cmd} (Mac)
					\item \textbf{Win (Windows)} $\rightarrow$ \textbf{Windows}
					\item \textbf{Fn (Function)} $\rightarrow$ \textbf{Fn}
				\end{itemize}
			\end{block}
		\end{column}
		
		\begin{column}{0.32\textwidth}
			\begin{block}{\textbf{Navigation Keys}}
				\begin{itemize}
					\small
					\item \textbf{Arrow keys} $\rightarrow$ \textbf{setas direcionais}
					\item \textbf{Enter/Return} $\rightarrow$ \textbf{Enter}
					\item \textbf{Tab} $\rightarrow$ \textbf{Tab}
					\item \textbf{Escape (Esc)} $\rightarrow$ \textbf{Esc}
					\item \textbf{Backspace} $\rightarrow$ \textbf{Backspace}
					\item \textbf{Delete (Del)} $\rightarrow$ \textbf{Delete}
				\end{itemize}
			\end{block}
		\end{column}
		
		\begin{column}{0.32\textwidth}
			\begin{block}{\textbf{Function Keys}}
				\begin{itemize}
					\small
					\item \textbf{F1-F12} $\rightarrow$ \textbf{Teclas de função}
					\item \textbf{Print Screen (PrtSc)} $\rightarrow$ \textbf{Imprimir tela}
					\item \textbf{Scroll Lock} $\rightarrow$ \textbf{Scroll Lock}
					\item \textbf{Pause/Break} $\rightarrow$ \textbf{Pausa/Interromper}
					\item \textbf{Insert (Ins)} $\rightarrow$ \textbf{Insert}
					\item \textbf{Home/End} $\rightarrow$ \textbf{Home/End}
				\end{itemize}
			\end{block}
		\end{column}
	\end{columns}
	
	\vspace{0.5cm}
	
	\begin{alertblock}{\textbf{Note on Terminology}}
		In documentation: "\textbf{Hold Ctrl and press C}" = "\textbf{Mantenha Ctrl pressionado e pressione C}"
	\end{alertblock}
\end{frame}

\begin{frame}{Reading Keyboard Shortcuts}
	\begin{block}{\textbf{Common Notations in Documentation}}
		\begin{itemize}
			\item \textbf{Ctrl + C} $\rightarrow$ \textbf{Copy} (Copy)
			\item \textbf{Ctrl + V} $\rightarrow$ \textbf{Paste} (Colar)
			\item \textbf{Ctrl + Z} $\rightarrow$ \textbf{Undo} (Desfazer)
			\item \textbf{Ctrl + Y} $\rightarrow$ \textbf{Redo} (Refazer)
			\item \textbf{Ctrl + S} $\rightarrow$ \textbf{Save} (Salvar)
			\item \textbf{Ctrl + F} $\rightarrow$ \textbf{Find} (Localizar)
			\item \textbf{Ctrl + P} $\rightarrow$ \textbf{Print} (Imprimir)
			\item \textbf{Alt + Tab} $\rightarrow$ \textbf{Switch windows} (Alternar janelas)
		\end{itemize}
	\end{block}
\end{frame}    
\begin{frame}
	\begin{exampleblock}{\textbf{How to Read/Write Shortcuts}}
		\begin{tabular}{ll}
			\textbf{Notation:} & \textbf{Meaning:} \\
			\texttt{Ctrl + C} & ``Control plus C" or "Control-C" \\
			\texttt{Ctrl + Shift + S} & ``Control Shift S" \\
			\texttt{Alt + F4} & ``Alt F4" \\
			\texttt{Cmd + Q} & ``Command Q" (Mac) \\
			\texttt{Ctrl + Alt + Delete} & ``Control Alt Delete" \\
		\end{tabular}
	\end{exampleblock}
	
	\begin{center}
		\footnotesize
		\textbf{Important:} The \textbf{+} symbol means ``press together" or ``while holding"
	\end{center}
\end{frame}
\begin{frame}{Punctuation Symbols}
    \centering
    \begin{tabular}{cl}
	        \toprule
	        \textbf{Symbol} & \textbf{Name} \\
	        \midrule
	        ; & semicolon \\
	        : & colon \\
	        , & comma \\
	        . & period/dot \\
	        ! & exclamation mark \\
	        ? & question mark \\
	        " & quotation marks \\
	        ' & apostrophe \\
	        \bottomrule
	    \end{tabular}
\end{frame}

\begin{frame}{Symbols - Punctuation}
	\centering
	\scriptsize
	\begin{tabular}{cl}
		\toprule
		\textbf{Symbol} & \textbf{Name} \\
		\midrule
		| & vertical bar (pipe) \\
		\textbackslash & backslash \\
		/ & forward slash \\
		\_ & underscore \\
		< & less than \\
		> & greater than \\
		\& & ampersand \\
		\# & number sign (hash) \\
		\bottomrule
	\end{tabular}
\end{frame}

\begin{frame}{Symbols - Special Characters}
	\centering
	\scriptsize
	\begin{tabular}{cl}
		\toprule
		\textbf{Symbol} & \textbf{Name} \\
		\midrule
		@ & at sign \\
		\$ & dollar sign \\
		\% & percent sign \\
		= & equals sign \\
		! & exclamation mark \\
		? & question mark \\
		* & asterisk \\
		+ & plus sign \\
		- & hyphen/minus sign \\
		\bottomrule
	\end{tabular}
\end{frame}

\begin{frame}{Terminal}
	A computer terminal is an electronic or electromechanical hardware device that can be used for entering data into, and transcribing data from, a computer or a computing system. Most early computers only had a front panel to input or display bits and had to be connected to a terminal to print or input text through a keyboard.
\end{frame}

\begin{frame}{Dumb terminal}
	A terminal that depends on the host computer for its processing power is called a "dumb terminal"[3] or a thin client.[4][5] In the era of serial (RS-232) terminals there was a conflicting usage of the term ``smart terminal"~ as a dumb terminal with no user-accessible local computing power but a particularly rich set of control codes for manipulating the display; this conflict was not resolved before hardware serial terminals became obsolete.
\end{frame}

\begin{frame}{Dave's Old Computers - Resources}
	\begin{columns}

		\begin{column}{0.5\textwidth}
			\centering
			\includegraphics[scale=0.8]{daves-old-computers}
			\footnotesize
			\href{http://dunfield.classiccmp.org}{\texttt{http://dunfield.classiccmp.org}}
			
			
		\end{column}
		

		\begin{column}{0.5\textwidth}
			\centering
			

			\qrcode[height=1.3in]{http://dunfield.classiccmp.org}
			
			\vspace{0.3cm}
			
		\end{column}
	\end{columns}
	\vspace*{0.5cm}
	\begin{center}
		\footnotesize
		\textit{Dave's Old Computers} It is a reference file for classical computing enthusiasts.
	\end{center}
\end{frame}


\begin{frame}[allowframebreaks]{Referências Bibliográficas}
	\justifying % Justifica o texto
	\small % Reduz um pouco o tamanho da fonte
	
	\begin{thebibliography}{9}
		
		\bibitem{wikipedia-informatics}
		Wikipedia contributors. (2025, December 7). Informatics. In Wikipedia, The Free Encyclopedia. Retrieved 18:03, January 29, 2026, from \url{https://en.wikipedia.org/w/index.php?title=Informatics&oldid=1326169972}. Acesso em: 29 jan. 2026
		
	
		\bibitem{wikipedia-bios}
		LINGUAGEM DE PROGRAMAÇÃO. In: \textbf{Wikipédia, a enciclopédia livre}. Flórida: Wikimedia Foundation, 2025. Disponível em: \url{https://en.wikipedia.org/wiki/BIOS}. Acesso em: 29 jan. 2026.
		
%		\framebreak
		
		\bibitem{w3schools}
	Keyboard shortcuts. Disponível em: \url{https://www.w3schools.com/tags/ref_keyboardshortcuts.asp}. Acesso em: 29 jan. 2026.
	
	
		
		\framebreak
		
		\bibitem{microsoft-shortcuts}
		Keyboard shortcuts in Windows. Disponível em: \url{https://support.microsoft.com/en-us/windows/keyboard-shortcuts-in-windows-dcc61a57-8ff0-cffe-9796-cb9706c75eec}. Acesso em: 29 jan. 2026.
		
		
	\end{thebibliography}
\end{frame}	

%\end{frame}
\end{document}
