\documentclass{if-beamer}
\begin{document}
	
	\title[Campus Boca do Acre]{Inglês instrumental}
	\subtitle{Class 1 }
	\author{Gilmar Gomes do Nascimento}
	\institute[IFAM]{
		Instituto Federal do Amazonas\\
		Campus Boca do Acre
	}
	\date{\today}
	\logo{
		\includegraphics[scale=0.0085]{logo.jpg}
	}
	\subject{Inglês Instrumental} % metadata
	
	\graphicspath{{figuras/}}
	%---------------------------------------------------------------------------------
	\begin{frame}
		\titlepage
	\end{frame}
	
	%\begin{frame}{Formação Acadêmica}
	%\begin{itemize}
	%\item Ciência da Computação - UFT   2009 -- 2014
	%\item Especialização Lato Senso - ESAB - 2019
	%\item Especialização Lato Senso - Focus - 2021
	%\item Mestrado em Ciência da Computação - UTFPR - 2024 - atualmente 
	%\end{itemize}
	%
	%\end{frame}
	%
	%\begin{frame}{Atuação Profissional}
	%\begin{itemize}
	%\item Instrutor de Informática  - SENAC - TO  - 2014, 2015
	%\item Professor de Informática - Colégio da Polícia Militar - TO - 2015 - 2016
	%\item Analista de Suporte -  Justiça Federal  - TO - 2017 
	%\item Professor substituto EBTT - IFTO - Campus Colinas do Tocantins - 2017 - 2019
	%\item Professor EBTT - IFAM - Campus Boca do Acre - 2021 - atualmente
	%\end{itemize}
	%\end{frame}
	\begin{frame}{How to make a presentation? (tools)}
	\begin{itemize}
		\item Microsoft Power Point
		\item Libreoffice Impress
		\item Google Slides
		\item Canva
		\item Beamer(\LaTeX)
		\item Marp
	\end{itemize}
	\end{frame}
	
	\begin{frame}{Beamer}
		Beamer is a \LaTeX ~ document class for creating presentation slides, with a wide range of templates and a set of features for making slideshow effects.\vspace*{0.3in}
		
		It supports pdfLaTeX, LaTeX + dvips, LuaLaTeX and XeLaTeX. The name is taken from the German word ``Beamer" as a pseudo-anglicism for ``video projector". 
	\end{frame}
	
	\begin{frame}{Frames}
		Slides can be built up on-screen in stages as if by revealing text that was previously hidden or covered.\\ \vspace*{0.3in} This is handled with PDF output by creating successive pages that preserve the layout but add new elements, so that advancing to the next page in the PDF file appears to add something to the displayed page, when in fact it has merely redrawn the page. 
	\end{frame}
	
	\begin{frame}{Beamer Themes}
		
		\begin{columns}[T]
			\begin{column}{0.33\textwidth}
				\begin{itemize}
					\item \texttt{default}
					\item \texttt{AnnArbor}
					\item \texttt{Antibes}
					\item \texttt{Bergen}
					\item \texttt{Berkeley}
					\item \texttt{Berlin}
					\item \texttt{Boadilla}
					\item \texttt{CambridgeUS}
					\item \texttt{Copenhagen}
				\end{itemize}
			\end{column}
			
			\begin{column}{0.33\textwidth}
				\begin{itemize}
					\item \texttt{Darmstadt}
					\item \texttt{Dresden}
					\item \texttt{Frankfurt}
					\item \texttt{Goettingen}
					\item \texttt{Hannover}
					\item \texttt{Ilmenau}
					\item \texttt{JuanLesPins}
					\item \texttt{Luebeck}
					\item \texttt{Madrid}
				\end{itemize}
			\end{column}
			
			\begin{column}{0.33\textwidth}
				\begin{itemize}
					\item \texttt{Malmoe}
					\item \texttt{Marburg}
					\item \texttt{Montpellier}
					\item \texttt{PaloAlto}
					\item \texttt{Pittsburgh}
					\item \texttt{Rochester}
					\item \texttt{Singapore}
					\item \texttt{Szeged}
					\item \texttt{Warsaw}
				\end{itemize}
			\end{column}
		\end{columns}
		
	\end{frame}
	\begin{frame}{How to use}
			A theme in Beamer can be set using the command \texttt{\textbackslash usetheme\{themeName\}}.
		
		\vspace{0.3cm}
		
		There are 27 built-in themes in Beamer:
		
		\vspace{0.2cm}
		
		\begin{block}{Example Usage}
			\centering
			\texttt{\textbackslash usetheme\{Berlin\}} \quad or \quad \texttt{\textbackslash usetheme\{Singapore\}}
		\end{block}
	\end{frame}
	

	\begin{frame}{}
		\includegraphics[width=1.0\linewidth]{./marp/0}
	\end{frame}
	\begin{frame}
		\centering
		\includegraphics[width=1.0\linewidth]{./marp/1}
	\end{frame}
	\begin{frame}
		\centering
		\includegraphics[width=0.9\linewidth]{./marp/2}
	\end{frame}
	\begin{frame}
		\centering
		\includegraphics[width=0.8\linewidth]{./marp/3}
	\end{frame}
	\begin{frame}
		\centering
		\includegraphics[width=0.9\linewidth]{./marp/4}
	\end{frame}
	\begin{frame}
		\centering
		\includegraphics[width=0.8\linewidth]{./marp/5}
	\end{frame}
	\begin{frame}
		\centering
		\includegraphics[width=0.8\linewidth]{./marp/6}
	\end{frame}
	\begin{frame}
		\centering
		\includegraphics[width=0.8\linewidth]{./marp/7}
	\end{frame}
	\begin{frame}
		\centering
		\includegraphics[width=1.0\linewidth]{./marp/8}
	\end{frame}	
	\begin{frame}
		\centering
		\includegraphics[width=1.0\linewidth]{./marp/9}
	\end{frame}
	\begin{frame}
		\centering
		\includegraphics[width=1.0\linewidth]{./marp/10}
	\end{frame}
	\begin{frame}{Anytype}
		\centering
		\includegraphics[width=0.5\linewidth]{./marp/anytype}
	\end{frame}
	\begin{frame}
		\centering
		\includegraphics[width=0.75\linewidth]{./marp/anytype1}
	\end{frame}
	\begin{frame}
		\centering
		\includegraphics[width=0.75\linewidth]{./marp/anytype2}
	\end{frame}
	\begin{frame}
		\centering
		\includegraphics[width=0.75\linewidth]{./marp/anytype3}
	\end{frame}
	\begin{frame}
		\centering
		\includegraphics[width=0.7\linewidth]{./marp/anytype4}
	\end{frame}
%	\begin{frame}{Anytype}
%		\includegraphics[width=0.75\linewidth]{./marp/anytype5}
%	\end{frame}

	\begin{frame}{You turn}
		\centering
		\Huge
		Here comes a new challenger!
	\end{frame}
	\begin{frame}{Choose a software and explain}
		\begin{columns}
			\begin{column}{0.5\textwidth}
				\begin{enumerate}
					\item Deskflow
					\item Microsoft Powertoys
					\item Docker
					\item Darktable
					\item Apostrophe
					\item GIMP
					\item Inkscape
					\item Foliate
					\item Blender
					\item Xournal++
					\item OBS Studio
					\item PDF Merger \& Splitter
				\end{enumerate}
			\end{column}
			
			\begin{column}{0.5\textwidth}
				\begin{enumerate}
					\setcounter{enumi}{12}
					\item VLC
					\item Wine
					\item Klavaro
					\item Steam
					\item Boxes
					\item Builder
					\item MAME
					\item DBeaver
					\item Trello
					\item OpenProject
					\item XAMPP
					\item ZSNES
				\end{enumerate}
			\end{column}
		\end{columns}
	\end{frame}
	
	\begin{frame}{Activity: Software Pitch Presentation}
		\centering
		
		Create a Google Slides presentation about your chosen software \\
			and share it with: \textbf{\texttt{gilmar.nascimento@ifam.edu.br}}
			
		\vspace{0.5cm}
		
		\begin{alertblock}{Important Details}
			\begin{itemize}
				\item \textbf{Value:} 1 point (grading activity)
				\item \textbf{Format:} Google Slides presentation
				\item \textbf{Language:} English required
				\item \textbf{Sharing:} Give "commenter" or "editor" access
			\end{itemize}
		\end{alertblock}
		
	\end{frame}
		\begin{frame}{Grading Criteria:}
		\begin{columns}[T]
			\begin{column}{0.45\textwidth}
				\begin{itemize}
					\item \textcolor{blue}{\textbf{Content clarity}} (25\%)
					\item \textcolor{green}{\textbf{English quality}} (25\%)
				\end{itemize}
			\end{column}
			\begin{column}{0.45\textwidth}
				\begin{itemize}
					\item \textcolor{purple}{\textbf{Visual design}} (25\%)
					\item \textcolor{orange}{\textbf{Persuasive argument}} (25\%)
				\end{itemize}
			\end{column}
		\end{columns}
		
		\vspace{1.0cm}
		\footnotesize
		\textit{Presentation should be 5-7 slides including screenshots and explanations}
	\end{frame}
	
	
%	\begin{frame}[allowframebreaks]{Referências Bibliográficas}
%		\justifying % Justifica o texto
%		\small % Reduz um pouco o tamanho da fonte
%		
%		\begin{thebibliography}{9}
%			
%			\bibitem{wikipedia-informatics}
%			Wikipedia contributors. (2025, December 7). Informatics. In Wikipedia, The Free Encyclopedia. Retrieved 18:03, January 29, 2026, from \url{https://en.wikipedia.org/w/index.php?title=Informatics&oldid=1326169972}. Acesso em: 29 jan. 2026
%			
%			%	\framebreak % Quebra para o próximo frame/slide
%			
%			\bibitem{wikipedia-linguagem}
%			LINGUAGEM DE PROGRAMAÇÃO. In: \textbf{Wikipédia, a enciclopédia livre}. Flórida: Wikimedia Foundation, 2025. Disponível em: \url{https://pt.wikipedia.org/w/index.php?title=Linguagem_de_programação&oldid=69620271}. Acesso em: 23 fev. 2025.
%			
%			%		\framebreak
%			
%			\bibitem{microsoft-extensoes}
%			MICROSOFT. \textbf{Extensões de nome de arquivo comuns no Windows}. Disponível em: \url{https://support.microsoft.com/pt-br/windows/extensões-de-nome-de-arquivo-comuns-no-windows-da4a4430-8e76-89c5-59f7-1cdbbc75cb01}. Acesso em: 7 ago. 2025.
%			
%			\framebreak
%			
%			\bibitem{fernandes-texto}
%			FERNANDES, M.; DIANA, D. \textbf{Texto Injuntivo}. Disponível em: \url{https://www.todamateria.com.br/texto-injuntivo/}. Acesso em: 7 ago. 2025.
%			
%			%	\framebreak
%			
%			\bibitem{mdn-javascript}
%			MDN. \textbf{JavaScript}. Disponível em: \url{https://developer.mozilla.org/en-US/docs/Web/JavaScript}. Acesso em: 7 ago. 2025.
%			
%			%	\framebreak
%			
%			\bibitem{rossini-anotacoes}
%			ROSSINI, M. C. \textbf{5 razões científicas para anotar coisas no papel em vez do celular}. Disponível em: \url{https://super.abril.com.br/ciencia/5-razoes-cientificas-para-anotar-coisas-no-papel-em-vez-do-celular/}. Acesso em: 7 ago. 2025.
%			
%		\end{thebibliography}
%	\end{frame}	
	
	%\end{frame}
\end{document}
